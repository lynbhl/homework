\documentclass{article}

\usepackage[english]{babel}
\usepackage[utf8]{inputenc}
\usepackage{amsmath,amssymb}
\usepackage{parskip}
\usepackage{graphicx}
\usepackage{enumerate}

% Margins
\usepackage[top=2.5cm, left=3cm, right=3cm, bottom=4.0cm]{geometry}
% Colour table cells
\usepackage[table]{xcolor}

% Get larger line spacing in table
\newcommand{\tablespace}{\\[1.25mm]}
\newcommand\Tstrut{\rule{0pt}{2.6ex}}         % = `top' strut
\newcommand\tstrut{\rule{0pt}{2.0ex}}         % = `top' strut
\newcommand\Bstrut{\rule[-0.9ex]{0pt}{0pt}}   % = `bottom' strut

%%%%%%%%%%%%%%%%%
%     Title     %
%%%%%%%%%%%%%%%%%
\title{MACHINE LEARNING 1}
\author{Bui Hoang Linh - 11205706}
\date{Homework 1}

\begin{document}
\maketitle

%%%%%%%%%%%%%%%%%
%   Problem 1   %
%%%%%%%%%%%%%%%%%
\section {Exercise 1}
a)
The marginal distribution p(x) and p(y):
\begin{itemize}
    \item p(x)
    \begin{align}
        p(x_1) = p(X=x_1) = 0.01 + 0.05 + 0.01 = 0.16\\
        p(x_2) = p(X=x_2) = 0.02 + 0.1 + 0.05 = 0.17\\
        p(x_3) = p(X=x_3) = 0.03 + 0.05 + 0.03 = 0.11\\
        p(x_4) = p(X=x_4) = 0.1 + 0.07 + 0.05 = 0.22\\
        p(x_5) = p(X=x_5) = 0.1 + 0.2 + 0.04 = 0.34
    \end{align}
    \item p(y)
    \begin{align}
        p(y_1) = p(Y=y_1) = 0.01 + 0.02 + 0.03 + 0.1 + 0.1 = 0.26\\
        p(y_2) = p(Y=y_2) = 0.05 + 0.1 + 0.05 + 0.07 + 0.2 = 0.47\\
        p(y_3) = p(Y=y_3) = = 0.1 + 0.05 + 0.03 + 0.05 + 0.04 = 0.27
    \end{align}
       
\end{itemize}
b)
The condition distribution p(x|Y=y1) and p(x|Y=y3):
\begin{itemize}
    \item $p(x|Y=y_1)$
    \begin{align}
        p(x_1|Y=y_1) = \frac{p(X=x_1,Y=y_1)}{p(y_1)} = \frac{0.01}{0.26} \approx 0.038\\
        p(x_2|Y=y_1) = \frac{p(X=x_2,Y=y_1)}{p(y_1)} = \frac{0.02}{0.26} \approx 0.077\\
        p(x_3|Y=y_1) = \frac{p(X=x_3,Y=y_1)}{p(y_1)} = \frac{0.03}{0.26} \approx 0.115\\
        p(x_4|Y=y_1) = \frac{p(X=x_4,Y=y_1)}{p(y_1)} = \frac{0.1}{0.26} \approx 0.385\\
        p(x_5|Y=y_1) = \frac{p(X=x_5,Y=y_1)}{p(y_1)} = \frac{0.1}{0.26} \approx 0.385\\
    \end{align}
    \item $p(x|Y=y_3)$
    \begin{align}
        p(x_1|Y=y_3) = \frac{p(X=x_1,Y=y_3)}{p(y_3)} = \frac{0.1}{0.27} \approx 0.37\\
        p(x_2|Y=y_3) = \frac{p(X=x_2,Y=y_3)}{p(y_3)} = \frac{0.05}{0.27} \approx 0.185\\
        p(x_3|Y=y_3) = \frac{p(X=x_3,Y=y_3)}{p(y_3)} = \frac{0.03}{0.27} \approx 0.11\\
        p(x_4|Y=y_3) = \frac{p(X=x_4,Y=y_3)}{p(y_3)} = \frac{0.05}{0.27} \approx 0.185\\
        p(x_5|Y=y_3) = \frac{p(X=x_5,Y=y_3)}{p(y_3)} = \frac{0.04}{0.27} \approx 0.148\\
    \end{align}
\end{itemize}


%%%%%%%%%%%%%%%%%
%   Problem 2   %
%%%%%%%%%%%%%%%%%
\section{Exercise 2}
We have:
\begin{eqnarray*}
    E_Y[E_X[x|y]] & = & \sum E(X|Y=y) \cdot P(Y=y) \\
    & = & \sum_y \sum_x x P(X=x|Y=y) \cdot P(Y=y) \\
    & = & \sum_y \sum_x x P(Y=y|X=x) \cdot P(X=x) \\
    & = & \sum_x P(X=x) \sum_y P(Y=y|X=x) \\
    & = & E_X(X)
\end{eqnarray*}

%%%%%%%%%%%%%%%%%
%   Problem 3   %
%%%%%%%%%%%%%%%%%

\section{Exercise 3}
Set variable names: \begin{itemize}
    \item X: people who use product X
    \item Y: people who use product Y
\end{itemize}
Then, we have:\\
P(X)=0.207   ,  P(Y)=0.5   ,  P(A|B)=0.365
\begin{enumerate}[a)]
    \item $P(AB) = P(A|B)\cdot P(B) = 0.5 \cdot 0.365 = 0.1825$
    \item $P(B|\overline{A}) = \frac{P(\overline{A}|B) \cdot P(B)}{P(\overline{A})} = \frac{(1-0.365) \cdot 0.5}{1 - 0.207} = 0.4004$
\end{enumerate}



 %%%%%%%%%%%%%%%%%
%   Problem 4   %
%%%%%%%%%%%%%%%%%

\section{Exercise 4}
Standard deviation:
$V_X[x] = E_X[(x-\mu)^2]$\\
Using the properties of avarage we have:\\
\begin{eqnarray*}
V_X[x] & = & E_X[(x-\mu)^2] \\
& = & E_X[x^2 - 2x\mu + \mu^2]\\
& = & E_X[x^2] - E_X[2x\mu] + E_X[\mu^2]\\
& = & E_X[x^2] -2\mu E_X[x] + \mu^2\\                
& = & E_X[x^2] -2\mu^2 + \mu^2\\
& = & E_X[x^2] - \mu^2
\end{eqnarray*}
We have: $\mu = E_X[x]$ \\
 $ => V_X[x]  =  E_X[(x-\mu)^2]  =  E_X[x^2] -(E_X[x])^2 $

%%%%%%%%%%%%%%%%%
%   Problem 5   %
%%%%%%%%%%%%%%%%%

\section{Exercise 5}
Set variable names:
\begin{itemize}
    \item A: door 1 has a car
    \item B: Monty open the door 2
    \item C: door 3 has a car
\end{itemize}
$P(A) = \frac{1}{3}$\\
$P(B) = \frac{1}{2}$\\
We have: $P(B|A) = \frac{1}{2}$ is the probability Monty open the door 2 given that the door 1 has the car\\
$P(A|B) = \frac{1}{3}$ is the probability that the car is in the door 1 after Monty open the door 2\\
Event A and event C are 2 mutually exclusive events so:\\
$=> P(C) = 1 - P(A) = \frac{2}{3}$\\
$=>$ The chances of winning the car are indeed 2 times higher (2/3) when you switch than when you stick (1/3)




\end{document}